\documentclass[10pt,a4paper]{book}

\title{Embrace}
\author{Marco Ferra}
\date{2019-2020 Lisbon - London}

\hyphenation{prob-lems con-scious-ness un-der-stand ex-pec-ta-tions mile-stones con-tin-u-ous}

\begin{document}

\maketitle

\tableofcontents

\chapter{Lead Authentically}

\section{Leave your ego at the door.}

External factors like other people or circumstances don't cause problems. Problems stem from our attitudes, selfishness, and self-absorption. You are correct if, after reading the last sentence, you think it's an oversimplification and that some problems originate externally. Nonetheless, it is pretentious to think that we have control over external events, even if we sometimes feel we have control over them. On top of that, people tend to have loss aversion - given the same conditions, there is a natural tendency to prefer avoiding losses to acquiring equivalent gains. So, every day, and especially when things get rough and unconformable, we must embrace humility, lead authentically, and have confidence in our abilities. Create yourself and be yourself.

\section{Speak up and make the tough calls.}

Finding the proper words to describe what we have in our minds is difficult. Perhaps we are rude. Perhaps we are overreacting. We may have not understood the situation well enough. The truth is that we are always overly harsh or polite. We are always over-reacting or under-reacting. If you believe you can fully understand and act perfectly according to the situation, be prepared to be frustrated and disappointed. No one can do so. So, the next logical step is to speak up and learn from your actions and reactions. If you speak up and act, you have accomplished more and given more of yourself to society. In the long run, and in the end, everyone benefits.

\section{Learn from your mistakes.}

You are accountable for yourself, not others. We continuously make, consciously or unconsciously, judgments about other people. It's how we compare, understand, and react accordingly. We blame others as an excuse for ourselves. We blame ourselves as an excuse for others. We see the mistake as the root cause, something that shouldn't have happened, an obstacle that shouldn't be there. However, it is when mistakes happen that people can learn. The unfortunate outcome that just happened is Nature's way of creating the proper environment for growth. Embrace the mistake, and if at first it's challenging to accept and celebrate it, at least see it as a learning opportunity. Smile and appreciate the mistake. It's pointless to blame others, and it doesn't change anything in the end. Take responsibility, take your conclusions, learn, and move on.

\section{Trust, integrity, respect and empathy.}

Trust is the confidence in or dependence on a person. Integrity is the practice of being honest and consistently adhering to strong moral and ethical principles and values. Everyone is different, and if someone's thoughts or actions seem odd, consider that perhaps you seem strange to someone else, too. Everyone thinks that they are correct, and most people have good intentions. To build meaningful relationships, you must create a proper environment. It takes consciousness and energy to do it. Choose your relationships. Sometimes things are easy to do, and sometimes they are hard, but if your relationship is strong enough based on these values, whatever happens, everyone benefits.

\chapter{Align People}

\section{Think strategically about the business and customers.}

A strategy is a high-level plan to achieve one or more goals under conditions of uncertainty, and an uncertain and unexpected environment is a familiar battleground for all of us. To think strategically is to be conscious of the surrounding environment and try to see the connections and needs, implied or explicit, of everyone around us. Everyone has the same 24 hours per day and has their own experiences, expectations, and difficulties. People change, adapt, and see reality in their way, in their own time. We are responsible for our behavior, and every action should be aligned with our mental picture, the environment, and the common good. What we think, say, and do directly impacts the whole picture.

\section{Excite people around your vision.}

Different things motivate different people. Some people are more energetic and active. Others have less energy and are more passive. Some of us wait for something to happen. Others blame or rationalize events, trying to justify the undesired outcome. When something changes, the uncertainties become more evident, and the surrounding environment becomes uncomfortable. The bottom line is that it doesn't matter how people are and how they see the world. We should be able to gather people together and get them excited around a shared vision. Our motivation changes other people's behavior, and walking together makes the journey more comfortable and pleasant. Being the CEO of our own life allows us to change our own lives and the lives of others for the better. Be brave and take the opportunity to help them. Go together.

\section{Provide a clear context and direction.}

Our mind works like a muscle that adapts to anything. It can learn how to do something correctly and can also learn how to do something incorrectly. Our minds don't know the difference between the two: in this regard, it works just like an unconscious machine, a mighty and dangerous one. However, we do have consciousness, an intrinsic awareness, and we should construct an inner framework that allows us clarity. That clear vision establishes the proper environment to train the mind correctly. Given enough training, we should get a more unobstructed view and communicate our perspective to others more precisely. We must ensure that others can see what we are seeing because their goals and expectations can only be understood if we do. Communicate what you see, what you expect, and your ambitions. Pay close attention to others. If you align your vision with theirs, everyone can go in the same direction.

\section{Bring the right people together.}

A problem is a situation that needs attention and needs to be dealt with or solved. Problems are what make the world go forward. Without solving problems, our life and existence would halt and, inevitably, collapse. The problem is not the problem itself (no pun intended!). The challenge is to gather the right people with the proper skill set to solve the problem. People can have differing views on the solution even if they have a shared vision. Getting the right people is getting valuable and different perspectives and managing those other points of view to arrive at a joint solution. You should bring people together because, in the end, everyone benefits. You can be that central cohesion piece of the puzzle.

\chapter{Grow Incredible Talent}

\section{Create opportunities for people to learn and grow.}

People grow in the proper environment. It's a natural process, and stopping it is impossible. It's more complicated and resource-consuming to stop the growth than to let it happen. The challenge is to create the right environment to learn and grow faster. People want to be part of something, they want to understand, and they are intrinsically good. Being the CEO of our own life comprehends creating opportunities for others. When others thrive, we also succeed. If they break something, help them make things right. If they are afraid or have concerns, go with them. If they are sad, listen and support them. If you are concerned with something, tell them. New opportunities flourish if you establish an active and supportive community around you. Be the one who creates that environment.

\section{Build high-performing teams.}

The Olympic Games were a series of athletic competitions among representatives of city-states of ancient Greece. You already have an informal team: everyone around you, family, friends, acquaintances, and strangers. You may even have a formal team, everyone that reports to you. You are the coach of those teams, and you are there to build high-performing athletes. Having and achieving high standards within a diverse team is no easy endeavor. Anyone can easily be a hero in their local setting, but competing in a World-Class environment with a top performance takes work. Building solid teams requires discipline, resilience, and aiming for the common good. They can get better, and you can get stronger, but only if you give yourself a chance to work on it. Bring people together and exercise your muscles every day.

\section{Give clear feedback.}

To be underperforming is to underachieve, not to reach standards or expectations. Through persistence and effort, teams grow and attain new development milestones. But there was a point in time when they were underperforming. Every learning and growth journey encompasses the cycle of underperforming and performing until the next challenge. As the coach of yourself and others, you should give clear feedback and help the teams overcome the obstacles. If you wait too much, they and you can't improve. As soon as you see an opportunity for growth, tell everyone about it. Set the expectations for yourself and your team, and give a clear message about it. When you see the team struggling, help them immediately and gather their feedback to adjust your response. Keep the communication channels open and establish the pace.

\section{Seek opportunities to coach your team.}

Coaching is a form of development in which an experienced person, called a coach, supports a learner or client in achieving a specific personal or professional goal by providing training and guidance. It's only possible to coach when there's space for improvement, which occurs when an opportunity arises. Pure chance can create opportunities, but also can people, and people can create opportunities much faster than pure luck. Set the environment, create the opportunities, and spotlight your team. Start coaching them by understanding them, taking note of their assumptions and goals, and providing guidance. You don't have all the answers, so walk together, be accountable for their growth, and be the coach of a fantastic team. Be the last to leave the building and the first to start the day.

\section{Celebrate and reward the achievements of everyone.}

The achievement of your team is your achievement. Your team achievement is the deliverable that enables you to grow as the CEO of your life. It's only a matter of time until you get the desired results. Some days, you'll feel tired. Others, you'll be immersed in a never-ending fog without a clear picture of where to go. Others feel the same, and occasionally, you'll bump into each other. It's all part of the journey until, perhaps without expecting, you and your team reach the goal. As soon as you get there, celebrate. Stop everything and cheer and reward achievement. All anyone asks for is a chance to work with pride. Everyone who puts energy and time into something desires celebration and recognition, and nothing is more beautiful than a smile and a handshake of "Well done!".

\chapter{Drive Great Outcomes}

\section{Deliver exceptional outcomes.}

If we could establish an imaginary line between the quality of outcomes that people deliver and another line with their true potential, we would see a gigantic gap between the two. Most of that gap comes from a mere lack of awareness. Our body has more strength than we think, and our minds can comprehend more things than we believe. We only need to know the desired results to deliver exceptional outcomes and focus our time and energy on achieving them. As soon as we find ourselves doing the same routine and feeling that we are not having what we would call an exceptional result, we should stop, be aware of it, and start again. Do this every day, and while doing so, show others how you do it. Show them the benefits, exemplify awareness, and expect nothing less than genuinely brilliant results. Be aware and be positive with everyone around you.

\section{Make bold decisions.}

Time is the indefinite continued progress of existence and events that occur in apparently irreversible succession from the past, through the present, to the future. Perhaps an overly complicated statement to tell the obvious - time is counting. Waiting for something to happen is putting ourselves in the hands of fate or the hands of others. Turn into a daily habit of making conscious decisions about everything. Be ambitious when doing so. Create challenges for yourself and accept the challenges that others give to you. Every opportunity arrives at the perfect time to exercise the skill of decision-making. Sometimes we make the wrong decision. Other times, you'll wonder if you could have done it another way. Be serious-minded and analyze what you could have done better. Learn from it and move forward. Doing this pushes you to uncharted territories, and you can create more opportunities for success. Keep moving forward.

\section{Voice your opinion and listen entirely.}

It's hard to find and discuss facts, but it's easy to discuss opinions. Even more, if they carry emotional content and people heartedly believe in that specific point of view. All ideas are rooted in subjective matters without conclusive findings. By figures of authority, by role models, or by just bullying, you may feel more comfortable not expressing your opinions. If you do this, the world loses. Perhaps the views of others are more distant from the truth than yours. Or their knowledge may not be as good as yours. Either way, if you don't express your view, everyone loses an opportunity for growth. So, speak up, and always listen back. Encourage others to do the same and try to reach an understanding of the problem. Even if, in the end, you don't arrive at a definite conclusion, the world can still be a better place. Don't stay in silence.

\section{Set ambitious goals.}

Every time you change a rule, you get a different result. So, think before complaining about the results. Being ambitious requires clear expectations for yourself and others. Everyone sees things from a different perspective. Find common ground to lay out the rules and be bold about the results you and your team should accomplish. Tell them how you view the path to those results. Listen to what they say. Adjust your view and define the rules as a team. Show that you are the first to abide by those rules. If you remember the legendary King Arthur's Round Table, you'll know that the table has no head, implying that everyone there has equal status. There is no king seating at the table. Establish the measurement system, discuss it, be accountable, and set ambitious goals for you and your team.

\chapter{First Principles Innovation}

\section{Drive game-changing innovation and continuous improvement.}

Everyone wants to feel special, and at the same time, everyone wants to belong to something, to some group. So how can we be special, intrinsically different from everything else, and simultaneously be in a group of common traits? Change the game. Be unique in your group. Be different and turn new ideas into new actions. Try and fail. Try again. Have a beginner's mind where every day is a unique opportunity to learn, improve, and go a step further. Do this with others, your family, friends, and team. Instill a continuous improvement mindset until it becomes completely automatic. If you do this, you belong to a minority group of innovators, game changers, and creative leaders. Those who initiate change can have a better opportunity to manage the transition. It's inevitable. Be that force.

\section{Show deep curiosity.}

People love to be correct. Often, they are not. People feel that their business is unique, different, and special from others. Often, it's not. Stay curious and learn that there's more than one way of doing things, much better ways. There are interested people everywhere, and perhaps the challenge you have in front of you already has been solved by someone else. Create informal groups, establish personal connections, and get along with other curious people like yourself. Inquisitive people shine everywhere and glow when they make a discovery. Gather your team, show a deep curiosity about their work, and implement what you have learned. Be like a toddler learning how to walk.

\section{Turn ideas into pragmatic, customer focused solutions.}

If you think that you exist for yourself, you would be mistaken. Your customer is the reason why you exist. You may be brilliant, or your team may come up with excellent ideas, but if you turn those ideas into something the customer can understand, grab and value, it's only a good use of time and talent. The customer is the center of everything you and your team do. So be bold, listen carefully, and give the customer the best experience that you can have. Value the customer and enable him to be the CEO of his own life. He'll thank you, and you'll thank him.

\section{Help others adapt to change.}

I hope that this book on these short essays enables change in your life. Keep it around, show it to others, and turn it into your actions that you have read here. Everyone feels discomfort when facing change, and so do you. However, now, you are better prepared. So, start now, help others, and be the real CEO that every company should have, and every employee should have as a leader. Lead yourself to the best you can do and enable others to do the same. Be kind and keep moving forward. You are already there. Thank you, and be the best you can be.

\end{document}